\usepackage{amsthm}
\usepackage{amsmath}
\usepackage{amsfonts}% to get the \mathbb alphabet
\usepackage{physics}
\usepackage{xcolor}
\usepackage[colorlinks=true, linkcolor=blue, citecolor=blue, urlcolor=blue]{hyperref}
\usepackage{orcidlink}
\usepackage{mdframed}
\usepackage{natbib}
\bibliographystyle{mnras}
\usepackage{tikz}
\usepackage{enumitem}
%%%%%% author info %%%%%%
\newcommand{\siyizhao}{\author{Siyi Zhao\,\orcidlink{0009-0001-4492-5158}}}
%%%%%% define some symbols %%%%%%
\newcommand{\field}[1]{\mathbb{#1}}
\newcommand{\C}{\field{C}}
\newcommand{\R}{\field{R}}
\newcommand{\Z}{\field{Z}}
\newcommand{\ii}{\mathrm{i}}
\newcommand{\vk}{\mathbf{k}}
\newcommand{\vx}{\vb{x}}
\newcommand{\diracd}{\delta^{\rm D}}
\newcommand{\diracdelta}[1]{\delta^{\rm D}_{#1}}
\newcommand{\assprime}[1]{{\left\langle#1\right\rangle}^\prime}
\newcommand{\cyc}[1]{#1\,\text{cyc.}}
\newcommand{\FT}{\quad \xrightarrow{\mathrm{FT}} \quad }
\newcommand{\vvr}{\mathbf{r}}
\newcommand{\vE}{\mathbf{E}}
\newcommand{\kB}{k_{\mathrm{B}}} % Boltzmann constant
%%%%%%%%% alphabet %%%%%%%%%
\newcommand{\Dcal}{\mathcal{D}}
\newcommand{\Ecal}{\mathcal{E}}
\newcommand{\Hcal}{\mathcal{H}}
\newcommand{\Kcal}{\mathcal{K}}
\newcommand{\kcal}{\mathcal{k}}
\newcommand{\Lcal}{\mathcal{L}}
\newcommand{\Ucal}{\mathcal{U}}
\newcommand{\ucal}{\mathcal{u}}
\newcommand{\Vcal}{\mathcal{V}}
%%%%%%%%% reference %%%%%%%%%
\newcommand{\refeq}[1]{Eq.~\ref{eq:#1}}
\newcommand{\refsec}[1]{Sec.~\ref{sec:#1}}
\newcommand{\reffig}[1]{Fig.~\ref{fig:#1}}
%%%%%%%%% notations %%%%%%%%%
\newcommand{\highlight}[1]{\colorbox{yellow!50}{#1}}
\newcommand{\attention}{\highlight{Pay attention:}}
\newcommand{\widelyuse}{\highlight{widely used}}
\definecolor{lightgray}{gray}{0.9}
\newcommand{\smallsection}[1]{\par\noindent\colorbox{lightgray}{\textbf{#1}}}
\newcommand{\concept}[1]{\textit{#1}}
\newcommand{\core}[1]{\textbf{\emph{#1}}}
\newcommand{\linkout}[1]{\textcolor{green}{#1}}
\newcommand{\QQQ}[1]{\textcolor{green}{\textbf{Q:} #1}}
\newcommand{\comment}[1]{\colorbox{green!50}{#1}}
%%%%%%%%% env: supplement %%%%%%%%%
\newmdenv[backgroundcolor=gray!20,shadow=true,shadowsize=6pt,roundcorner=10pt]{supplement}
%%%%%%%%% env: important %%%%%%%%%
\newmdenv[backgroundcolor=yellow!20,hidealllines=true]{important}

\swapnumbers % 将编号放在题目前

%%%%%%% env: remark %%%%%%%
\newtheorem*{remark}{Remark}
\let\oldremark\remark 
\renewenvironment{remark}
  {\oldremark\normalfont}
  {}
%%%%%%% modify the Remark env %%%%%%%
%%%%%%% env: theorem %%%%%%%
\newtheorem*{theorem}{Theorem}
\let\oldtheorem\theorem
\renewenvironment{theorem}
  {\oldtheorem\normalfont}
  {}
%%%%%%% modify the Theorem env %%%%%%%
\theoremstyle{definition} 
\newcommand{\nameddefname}{}
\newtheorem{innercustomdef}{\nameddefname} 
\newenvironment{nameddef}[1]
  {\renewcommand{\nameddefname}{#1}\innercustomdef}
  {\endinnercustomdef}

  \theoremstyle{plain}
\newcommand{\namedthmname}{}
\newtheorem{innercustomthm}[innercustomdef]{\namedthmname} 
\newenvironment{namedthm}[1]
  {\renewcommand{\namedthmname}{#1}\innercustomthm}
  {\endinnercustomthm}

\newenvironment{question}
  {
    \begin{quote} % 环境开始时的格式
    \textbf{Q:} % 可选的标题或格式
  }
  {
    \end{quote} % 环境结束时的格式
  }

\newenvironment{answer}
  {
    \begin{quote} 
    \textbf{A:} 
  }
  {
    \end{quote} 
  }

\numberwithin{innercustomdef}{section}
