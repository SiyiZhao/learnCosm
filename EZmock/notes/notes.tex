\documentclass{article}
\usepackage[width=170mm, margin=1in]{geometry} 
\usepackage{amsthm}
\usepackage{amsmath}
\usepackage{amsfonts}% to get the \mathbb alphabet
\usepackage{physics}
\usepackage{xcolor}
\usepackage[colorlinks=true, linkcolor=blue, citecolor=blue, urlcolor=blue]{hyperref}
\usepackage{orcidlink}
\usepackage{mdframed}
\usepackage{natbib}
\bibliographystyle{mnras}
\usepackage{tikz}
\usepackage{enumitem}
%%%%%% author info %%%%%%
\newcommand{\siyizhao}{\author{Siyi Zhao\,\orcidlink{0009-0001-4492-5158}}}
%%%%%% define some symbols %%%%%%
\newcommand{\field}[1]{\mathbb{#1}}
\newcommand{\C}{\field{C}}
\newcommand{\R}{\field{R}}
\newcommand{\Z}{\field{Z}}
\newcommand{\ii}{\mathrm{i}}
\newcommand{\vk}{\mathbf{k}}
\newcommand{\vx}{\vb{x}}
\newcommand{\diracd}{\delta^{\rm D}}
\newcommand{\diracdelta}[1]{\delta^{\rm D}_{#1}}
\newcommand{\assprime}[1]{{\left\langle#1\right\rangle}^\prime}
\newcommand{\cyc}[1]{#1\,\text{cyc.}}
\newcommand{\FT}{\quad \xrightarrow{\mathrm{FT}} \quad }
\newcommand{\vvr}{\mathbf{r}}
\newcommand{\vE}{\mathbf{E}}
\newcommand{\kB}{k_{\mathrm{B}}} % Boltzmann constant
%%%%%%%%% alphabet %%%%%%%%%
\newcommand{\Dcal}{\mathcal{D}}
\newcommand{\Ecal}{\mathcal{E}}
\newcommand{\Hcal}{\mathcal{H}}
\newcommand{\Kcal}{\mathcal{K}}
\newcommand{\kcal}{\mathcal{k}}
\newcommand{\Lcal}{\mathcal{L}}
\newcommand{\Ucal}{\mathcal{U}}
\newcommand{\ucal}{\mathcal{u}}
\newcommand{\Vcal}{\mathcal{V}}
%%%%%%%%% reference %%%%%%%%%
\newcommand{\refeq}[1]{Eq.~\ref{eq:#1}}
\newcommand{\refsec}[1]{Sec.~\ref{sec:#1}}
\newcommand{\reffig}[1]{Fig.~\ref{fig:#1}}
%%%%%%%%% notations %%%%%%%%%
\newcommand{\highlight}[1]{\colorbox{yellow!50}{#1}}
\newcommand{\attention}{\highlight{Pay attention:}}
\newcommand{\widelyuse}{\highlight{widely used}}
\definecolor{lightgray}{gray}{0.9}
\newcommand{\smallsection}[1]{\par\noindent\colorbox{lightgray}{\textbf{#1}}}
\newcommand{\concept}[1]{\textit{#1}}
\newcommand{\core}[1]{\textbf{\emph{#1}}}
\newcommand{\linkout}[1]{\textcolor{green}{#1}}
\newcommand{\QQQ}[1]{\textcolor{green}{\textbf{Q:} #1}}
\newcommand{\comment}[1]{\colorbox{green!50}{#1}}
%%%%%%%%% env: supplement %%%%%%%%%
\newmdenv[backgroundcolor=gray!20,shadow=true,shadowsize=6pt,roundcorner=10pt]{supplement}
%%%%%%%%% env: important %%%%%%%%%
\newmdenv[backgroundcolor=yellow!20,hidealllines=true]{important}

\swapnumbers % 将编号放在题目前

%%%%%%% env: remark %%%%%%%
\newtheorem*{remark}{Remark}
\let\oldremark\remark 
\renewenvironment{remark}
  {\oldremark\normalfont}
  {}
%%%%%%% modify the Remark env %%%%%%%
%%%%%%% env: theorem %%%%%%%
\newtheorem*{theorem}{Theorem}
\let\oldtheorem\theorem
\renewenvironment{theorem}
  {\oldtheorem\normalfont}
  {}
%%%%%%% modify the Theorem env %%%%%%%
\theoremstyle{definition} 
\newcommand{\nameddefname}{}
\newtheorem{innercustomdef}{\nameddefname} 
\newenvironment{nameddef}[1]
  {\renewcommand{\nameddefname}{#1}\innercustomdef}
  {\endinnercustomdef}

  \theoremstyle{plain}
\newcommand{\namedthmname}{}
\newtheorem{innercustomthm}[innercustomdef]{\namedthmname} 
\newenvironment{namedthm}[1]
  {\renewcommand{\namedthmname}{#1}\innercustomthm}
  {\endinnercustomthm}

\newenvironment{question}
  {
    \begin{quote} % 环境开始时的格式
    \textbf{Q:} % 可选的标题或格式
  }
  {
    \end{quote} % 环境结束时的格式
  }

\newenvironment{answer}
  {
    \begin{quote} 
    \textbf{A:} 
  }
  {
    \end{quote} 
  }

\numberwithin{innercustomdef}{section}


\title{EZmock}
\siyizhao
\date{\today}

\begin{document}
\maketitle
\tableofcontents

\section{Introduction}

\cite{zhao2021ezmock}

\section{Matter Field}

Zel'dovich approximation, the linear solution of the LPT.

The displacement field in ZA is given by
\begin{equation}
    \vec{\Psi}_{\mathrm{ZA}}(\vec{q}, a)=D_1(a) \int \frac{{\dd}^3 k}{{(2 \pi)}^3} \mathrm{e}^{\ii \vec{k} \cdot \vec{q}} \frac{\ii \vec{k}}{k^2} \hat{\delta}(\vec{k}),
\end{equation}
where $\hat{\delta}(\vec{k})$ is the density contrast in Fourier space at the initial time, and the linear growth factor $D_1(a)$ can be evaluated numerically through the integral representation
\begin{equation}
    D_1(a)=a^3 H(a) \frac{5 \Omega_{\mathrm{m}}}{2} \int_0^a \frac{\dd \tilde{a}}{\tilde{a}^3 H^3(\tilde{a})}.
\end{equation}


\section{Effective Bias Model}

Refer to Sec.2.2 of~\cite{zhao2021ezmock}.

\subsection{The Bias Function}

\begin{itemize}
    \item[1.] Critical density $\rho_{\rm c}$: To form gravitational bound systems, such as dark matter halos, \core{a minimum local denisty} is required to overcome the background expansion. The first free parameter is \core{a density threshold} $\rho_{\rm c}$, the bias model contains a term of step function $\Theta(\rho_{\mathrm{m}}-\rho_{\mathrm{c}})$ to set a density threshold, in ohter words, only grid cells with density equal or higher than $\rho_{\rm c}$ are allowed to populate tracers.
    \item[2.] Density saturation $\rho_{\rm sat}$: [TBD] Due to the strong degeneracies, they fix $\rho_{\rm sat}=10$.
    \item[3.] Exponential cut-off of the halo bias relation $\rho_{\exp}$: For centain $\rho_{\rm m}$, the larger $\rho_{\exp}$, the lower $\rho_{\rm t}$. When $\rho_{\rm m}\gg \rho_{\exp}$, this effect is very small, it mainly affects the low density regions. [TBD]
\end{itemize}

To summarize, the full bias model can be written as
\begin{equation}
    \rho_{\mathrm{t}}=\theta\left(\rho_{\mathrm{m}}-\rho_{\mathrm{c}}\right) \rho_{\mathrm{sat}}\left[1-\exp \left(-\rho_{\mathrm{m}} / \rho_{\exp}\right)\right] B_{\mathrm{s}},
\end{equation}
where $B_{\mathrm{s}}$ is the stochastic bias term to model the scatter in the bias relation.
\begin{equation}
    B_{\mathrm{s}}= \begin{cases}1+G(\lambda), & G(\lambda) \geq 0 ; \\ \exp (G(\lambda)), & G(\lambda)<0 .\end{cases}
\end{equation}
Here $G(\lambda)$ is a random number drawn from a Gaussian distribution with mean 0 and variance $\lambda$. 
The exponentail function is for ensuring the positivity of bias values.

The model of stochastic bias is too simple than people already known. The reason is, in EZmock, only the \core{order} of tracer densities for each cells matters.
They fixed $\lambda=10$ since the effect from it can be achieved by the other parameters, such as $\rho_{\rm c}$ and $\rho_{\rm exp}$.

\subsection{PDF Mapping Scheme}

Model the PDF of the tracers by a power-law relation
\begin{equation}
    P(n_{\rm t})=Ab^{n_{\rm t}},
\end{equation}
where $A>0$ and $0<b<1$.

The total tracer number is given by the reference cubic simulation, and it is 
\begin{equation}
    N_{\rm t}^{\rm tot}=\sum_{n_{\rm t}=1}^{n_{{\rm t}, \max}} n_{\rm c}(n_{\rm t}) n_{\rm t},
\end{equation}
where $n_{\rm c}(n_{\rm t}) = \left\lfloor N_{\text {cell }} P\left(n_{\mathrm{t}}\right)\right\rceil$ is the number of cells with tracer number $n_{\rm t}$, and $N_{\rm cell}$ is the total number of cells in the simulation box.

Then EZmock assign tracers to the cells according to the bias model in previous section. 
First it sort the cells by their tracer densities $\rho_{\rm t}$ in descending order.
Then it assign $n_{\rm c}(n_{\rm t})$ cells with the highest $\rho_{\rm t}$ to have $n_{\rm t}$ tracers, until all tracers are assigned.

\subsection{Random Motion for RSD}

Add an isotropic 3D Gaussian motion to the linear coherent velocity from ZA
\begin{equation}
    \vec{u}_{\rm t}=\vec{u}_{\mathrm{ZA}}+\vec{G}(v),
\end{equation}

\subsection{Summary}

To summarize, the effective bias model contains 4 free parameters:
\begin{itemize}
    \item $\rho_{\rm c}$: critical density threshold for tracer population.
    \item $\rho_{\exp}$: exponential cut-off controlling the low-density tracer population.
    \item $b$: parameter in the power-law PDF mapping scheme.
    \item $v$: amplitude of the random motion added for RSD.
\end{itemize}

\bibliography{EZmock}

\end{document}
